\documentclass[a4paper,12pt]{article}

\usepackage[english]{babel}
\usepackage{blindtext}
\usepackage[a4paper, inner=1.7cm, outer=2.7cm, top=2cm, bottom=2cm, bindingoffset=1.2cm]{geometry}
\usepackage{graphicx}
\usepackage{wrapfig}
\usepackage{enumitem}
\usepackage{fancyhdr}
\usepackage{amsmath}
\usepackage{index}
\usepackage{listings}
\usepackage{minted}

\newcommand{\torch}{\texttt{Pytorch}}
\begin{document}
\title{\Large{\textbf{Secure Collaborative Training of Machine Learning Model using MPC}}}
\author{By Souhail Meftah, Ruomu Hou}
\date{October 22, 2019}
\maketitle
\let\cleardoublepage\clearpage
%\tableofcontents
\setcounter{page}{1}
\fancyhf{}



\section{Background and Motivation}

\subsection{Multi-party Computation}

\subsection{iDash Privacy and Security Workshop 2019: Secure genome analytics competition}

\section{Methodology and Technologies}

\section{Implementation}
\subsection{Requirement and Setup}
\textbf{Notice: It requires at least a setup with 12GB memory to run the code.}

\begin{listing}[H]
	\caption{Setup the runtime environment}
	\inputminted[frame=single,framesep=10pt,linenos]{bash}{1_prep.sh}
	\label{code_setup}
\end{listing}

Our experiment runs on 1 machine on the Tembusu Cluster with Intel E5-2620V3, 256GB DDR3 RAM, and CentOS 7.x.x. The code has been tested on Ubuntu 18.04 as well. You could use the code listed in Listing \ref{code_setup} to run in similar environments.  

Alternatively, you may run the prepared docker image with\\
\texttt{sudo docker run -it --net=host houruomu/cs6203 jupyter notebook --allow-root}

\subsection{Load the Data}
\begin{listing}[H]
	\caption{Load the data}
	\inputminted[frame=single,framesep=10pt,linenos]{python3}{load_data.py}
	\label{code_load}
\end{listing}
Our code snippet in Listing \ref{code_load} are the instructions that we used to load the data from the text document. In the text file, the first row and first column are data labels. Each column of the file corresponds to a data sample with rows being the SNPs values (i.e. the features). The code prepares the data into 2d numpy arrays with each row corresponding to a sample.

\subsection{Model Creation}
\begin{listing}[H]
	\caption{Define the model}
	\inputminted[frame=single,framesep=10pt,linenos]{python3}{torch_model.py}
	\label{code_model}
\end{listing}
Our model is adapted from a well-estabilished convolutional model for analyzing temporal data. The neural network has two data flows, on one flow there is only a fully connected layer with 64 outputs, and on the other flow a convolutional sub-network is repeatedly applied to generate a rich feature space. Then the two data flows are concatenated and 2 fully connected layers are used to get the binary output.

The code snippet in Listing \ref{code_model} illustrates our model definition. In \torch, the networks are defined as classes where in the \texttt{\_\_init\_\_} method the layers are defined and in \texttt{forward} method the operations to compose the layers are defined. The subnet class \texttt{Res1d} is defined with parameters \texttt{inSize, outSize, kernel, strides} to be used by the \texttt{Net} class to instatiate subnets in the layers definition. Notice that the \texttt{Net} class has a hidden parameter \texttt{dim} declared as a global variable, which corresponds to the feature space dimension of the input data.

\subsection{Model Training}
\begin{listing}[H]
	\caption{Train the model}
	\inputminted[frame=single,framesep=10pt,linenos]{python3}{train.py}
	\label{code_train}
\end{listing}
We use mini-batched stochastic gradient descent with binary cross entropy loss function to train the parameters. Based on experimental results, we choose a batch size of 30, learning rate of 0.001 and momentum of 0.9. The code for training is in Listing \ref{code_train}.


\end{document}